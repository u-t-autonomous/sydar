%%%%%%%%%%%%%%%%%%%%%%%%%%%%%%%%%%%%%%%%%%%%%%%%%%%%%%%%%%%%%%%%%%%%%%%%%%%%%%%%
%2345678901234567890123456789012345678901234567890123456789012345678901234567890
%        1         2         3         4         5         6         7         8

\documentclass[letterpaper, 10 pt, conference]{ieeeconf}  % Comment this line out
                                                          % if you need a4paper
%\documentclass[a4paper, 10pt, conference]{ieeeconf}      % Use this line for a4
                                                          % paper

\IEEEoverridecommandlockouts                              % This command is only
                                                          % needed if you want to
                                                          % use the \thanks command
\overrideIEEEmargins
% See the \addtolength command later in the file to balance the column lengths
% on the last page of the document



% The following packages can be found on http:\\www.ctan.org
%\usepackage{graphics} % for pdf, bitmapped graphics files
%\usepackage{epsfig} % for postscript graphics files
%\usepackage{mathptmx} % assumes new font selection scheme installed
%\usepackage{times} % assumes new font selection scheme installed
\usepackage{amsmath} % assumes amsmath package installed
\usepackage{amssymb}  % assumes amsmath package installed

\title{\LARGE \bf
LTLOC
}


\author{Ivan Papusha - Mohammed Alshiekh - Ufuk Topcu% <-
}


\begin{document}



\maketitle
\thispagestyle{empty}
\pagestyle{empty}


\begin{abstract}
This paper describes LTLOC, a Python-based Domain Specific Language for the synthesis of optimal controllers for continuous-time and continuous-state systems under regular expression specifications. LTLOC employs a novel approach that avoids explicit discretization in synthesis.
\end{abstract}

\section{Input File Syntax and Parsing}

The result of parsing the input file is a symbol table with scopes corresponding to each block in the input file. Here we describe each of these blocks.

\subsection{Constants}
\subsection{Specification}
\subsection{Regions}
\subsection{APs}
\subsection{Letters}
\subsection{System}

\section{Regions and their representations}

Regions must be represented as either ellipsoids, half-spaces, empty or a combination of them represented as a tree with union and intersection operations.

\subsection{Ellipsoids}
(formal definitions of the representations)
\begin{itemize}
	\item $\mathcal{E} = \{ v \mid \|A v + b\|_2 \leq 1  \}$
	\item $\mathcal{E} = \{ x \mid x^T A x + 2 b^Tx + c \leq 0 \}$
	\item $\mathcal{E} = \{ x \mid (x - x_c)^T B^{-1} (x - x_c) \leq 1 \}$
\end{itemize}

\subsection{Halfspaces}
\begin{itemize}
	\item $\mathcal{H} = \{ x \mid 2b^Tx + c \leq 0 \}$
\end{itemize}
(formal definitions of the representations)

\subsection{Empty regions}
\begin{itemize}
	\item $\emptyset$
\end{itemize}

\section{Specifications and Regular Expressions}

Regular expressions are used as the specification language.

\section{Creating a Tagged Graph from Specification}

First we create DFA from the specification then we create a tagged graph from the DFA and APs and their regions.

\subsection{Graph Tagging}
The vertices must be tagged with an objective and a region tree. The edges must be tagged with a region tree.
 
\subsubsection{Vertex Tagging}
For each vertex, get the APs corresponding to the letter then intersect the AP's corresponding regions and tag the vertex with the resulting tree region.
\subsubsection{Edge Tagging}
For each edge, get the APs corresponding to the letter then intersect the AP's corresponding regions.
\section{Canonical 2-level Region Tree}

(Describe the tree and motive)

\section{Tree transformation}
The goal of the transformation is to transform any region tree to the canonical 2-level region tree. The tree first goes through tree preparation, then apply lossless tree transformations, if the tree turns out to be a canonical 2-level region tree after this step, then the transformation is completed. Else, we will have to apply lossy transformations.
\subsection{Tree Preparation}
This step adds a union node at the top with all intersection operands. The rest of the original tree starts after the 2nd level.
\subsection{Lossless Tree Transformation}

\subsubsection{Empty Region}

Empty region in intersection:
The whole subtree will be transformed to an empty region

Empty region in union:
Remove the empty region
		
		
\subsubsection{Single Leaf Nodes}
Remove the operation node and raise the terminal node a level.
	
\subsubsection{Redundant Operation Nodes}
Move the operands of the lower operation node to be included in the operands of the upper operation node.
\subsection{Lossy Tree Transformation}

\subsubsection{Union of two or more non-empty regions}
We wish to find a minimum volume (L\"owner--John) ellipsoid
$\mathcal{E}_\mathrm{lj}$ covering the union of ellipsoids and halfspaces. Given
\begin{align*}
	\mathcal{E}_i &= \{ x \mid x^T A_i x + 2 b_i^Tx + c_i \leq 0 \}, \quad i=1,\ldots,N\\
	\mathcal{H}_j &= \{ x \mid 2b_j^Tx + c_j \leq 0 \}, \quad j=1,\ldots,M.
\end{align*}
To find $\mathcal{E}_\mathrm{lj} = \{ x \mid \|Ax + b\|_2 \leq 1 \}$ that satisfies
\[
	\mathcal{E}_1 \cup \cdots \cup \mathcal{E}_N
	\cup
	\mathcal{H}_1 \cup \cdots \cup \mathcal{H}_M
	\subseteq
	\mathcal{E}_\mathrm{lj},
\]
we solve the problem by performing procedure
\begin{enumerate}
	\item Reformat the halfspaces $\mathcal{H}_1,\ldots,\mathcal{H}_M$ as
		ellipsoids with $A_j = 0$.
	\item solve the problem
	\[
		\mathcal{E}_1 \cup \cdots \cup \mathcal{E}_N
		\cup
		\mathcal{E}_{N+1} \cup \cdots \cup \mathcal{E}_{N+M}
		\subseteq
		\mathcal{E}_\mathrm{lj}.
	\]
	by the SDP~\cite[\S{}8.4.1]{Boyd:2004}
	\[
		\begin{array}{ll}
			\mbox{minimize}   & \log \det X^{-1/2} \\
			\mbox{subject to} 
				& \tau_1 \geq 0, \ldots, \tau_{N+M} \geq 0,\\
				& \begin{bmatrix}
					X - \tau_i A_i & \tilde{b} - \tau_i b_i & 0\\
					(\tilde{b} - \tau_i b_i)^T & -1 - \tau_i c_i & {\tilde{b}}^T \\
					0 & {\tilde{b}} & -X
				\end{bmatrix}
				\preceq 0, \\
				& \qquad i=1,\ldots,N+M
		\end{array}
	\]
	over the variables $X \in \mathbf{S}^n$, $\tilde{b} \in \mathbf{R}^n$,
	$\tau_1,\ldots,\tau_{N+M} \in \mathbf{R}$.
	\item The L\"owner--John ellipsoid is given by 
		$\mathcal{E}_\mathrm{lj} = \{ x \mid \|Ax + b\|_2 \leq 1 \}$, where
		\begin{align*}
			A &:= (X^\star)^{1/2}\\
			b &:= A^{-1}{\tilde{b}}^\star \quad (= (X^\star)^{-1/2} {\tilde{b}}^\star)
		\end{align*}
\end{enumerate}





\subsubsection{Intersection of two or more non-empty regions}
How?

\section{Matlab Code Generation}
We generate the matlab code from the set of nodes and edges tagged with objectives and 2-level canonical trees. We can breakdown the generation of MATLAB code into the following subsections.

\subsection{Constants Code Generation} 
\subsection{Objective Function Code Generation} 
\subsection{Constraints Code Generation}


%%%%%%%%%%%%%%%%%%%%%%%%%%%%%%%%%%%%%%%%%%%%%%%%%%%%%%%%%%%%%%%%%%%%%%%%%%%%%%%%
% Bibliography
%%%%%%%%%%%%%%%%%%%%%%%%%%%%%%%%%%%%%%%%%%%%%%%%%%%%%%%%%%%%%%%%%%%%%%%%%%%%%%%%
\bibliographystyle{abbrv}
\bibliography{refs}

\end{document}
